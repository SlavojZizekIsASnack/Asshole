\documentclass[12pt]{article}
\usepackage[danish]{babel}
\usepackage[utf8]{inputenc}

\begin{document}

\begin{titlepage}

	\newcommand{\HRule}{\rule{\linewidth}{0.5mm}}

	\center

	%---------------------------------------------------------------------------
	%	HEADING SECTIONS
	%---------------------------------------------------------------------------

	\textsc{\LARGE Learnmark Horsens}\\[1.5cm] % Name of your university/college
	\textsc{\Large Programering C}\\[0.5cm] % Major heading such as course name
	\textsc{\large 2. MFI}\\[0.5cm] % Minor heading such as course title

	%---------------------------------------------------------------------------
	%	TITLE SECTION
	%---------------------------------------------------------------------------

	\HRule \\[0.4cm]
	{ \huge \bfseries R\o vhul}\\[0.4cm]
	\HRule \\[1.5cm]

	%---------------------------------------------------------------------------
	%	AUTHOR SECTION
	%---------------------------------------------------------------------------

	\Large \emph{Skrevet af:}\\
	Konrad \textsc{Christensen}\\
	Markus \textsc{Langfeldt}\\[3cm]

	\vfill

\end{titlepage}

\pagebreak

%-------------------------------------------------------------------------------
%	TOC SECTION
%-------------------------------------------------------------------------------

\tableofcontents

\vfill

\pagebreak

%-------------------------------------------------------------------------------
%	MAIN SECTION
%-------------------------------------------------------------------------------

\section{Abstract}

Igennem denne opgave har vi arbejdet med at lave kortspillet "Røvhul", samt vi har arbejdet med at lave to AI's som er i stand til at spille spillet, på den mest optimale måde. Opgaven vil gå i gennem vores begrundelse for projektet i sektionen "Problemformulering", samt hvad vores plan at gøre med selve projektet,og hvordan vi i grove træk tænker det skal gøres.
\\
Derefter vil opgaven komme ind på hvordan de forskellige funktioner i programmerne fungere sammen med eksempler på hvordan de kan bruges. Der vil i kapitlet "Dokumentation", også blive snakket om hvordan projektet fungere, ud over de forskellige funktioner som bliver brugt. Efter følgende vil vi så konkludere om det er lykkes at opfylde vores krav til det fuldente program resultat, som vi satte som mål i problemformuleringen.

\section{Problemformulering} \label{sec:Problemformulering}

Der er mange forskellige strategier i kortspillet Roevhul. Ved at programmere forskellige AI med forskellige strategier kan vi spille dem mod hinanden og finde den bedste strategi.
\\
For at kunne dette skal vi foerst lave en implementering af Roevhul og derefter skal der identificeres og implementeres strategier. Vi forventer at vi med denne metode kan lave en AI der kan vinde over mennesker.

\vfill
\pagebreak

\section{Dokumentation}
\subsection{Rust}

\subsection{C++}


\vfill
\pagebreak

\section{Konklusion}

\vfill
\pagebreak

\end{document}

