\documentclass[a4paper, 12pt]{article}
\usepackage[danish]{babel}
\usepackage[utf8]{inputenc}
\usepackage{pdftexcmds}
\usepackage{minted}
\usepackage{tikz}
\usepackage{parskip}

\usetikzlibrary{shapes.geometric, arrows}


\begin{document}

\begin{titlepage}

	\newcommand{\HRule}{\rule{\linewidth}{0.5mm}}

	\center

	%---------------------------------------------------------------------------
	%	HEADING SECTIONS
	%---------------------------------------------------------------------------

	\textsc{\LARGE Learnmark Horsens}\\[1.5cm] % Name of your university/college
	\textsc{\Large Programering C}\\[0.5cm] % Major heading such as course name
	\textsc{\large 2. MFI}\\[0.5cm] % Minor heading such as course title

	%---------------------------------------------------------------------------
	%	TITLE SECTION
	%---------------------------------------------------------------------------

	\HRule \\[0.4cm]
	{ \huge \bfseries Røvhul}\\[0.4cm]
	\HRule \\[1.5cm]

	%---------------------------------------------------------------------------
	%	AUTHOR SECTION
	%---------------------------------------------------------------------------

	\Large \emph{Skrevet af:}\\
	Konrad \textsc{Christensen}\\
	Marcus \textsc{Langfeldt}\\
	Markus \textsc{Jakobsen}\\[3cm]

	\vfill

\end{titlepage}

\pagebreak

%-------------------------------------------------------------------------------
%	TOC SECTION
%-------------------------------------------------------------------------------

\tableofcontents

\vfill

\pagebreak

%-------------------------------------------------------------------------------
%	MAIN SECTION
%-------------------------------------------------------------------------------

\section{Abstract}

Igennem denne opgave har vi arbejdet med at lave kortspillet "Røvhul", samt vi har arbejdet med at lave to AI's som er i stand til at spille spillet, på den mest optimale måde. Opgaven vil gå i gennem vores begrundelse for projektet i sektionen "Problemformulering", samt hvad vores plan at gøre med selve projektet,og hvordan vi i grove træk tænker det skal gøres.
\bigbreak
Derefter vil opgaven komme ind på hvordan de forskellige funktioner i programmerne fungere sammen med eksempler på hvordan de kan bruges. Der vil i kapitlet "Dokumentation", også blive snakket om hvordan projektet fungere, ud over de forskellige funktioner som bliver brugt. Efter følgende vil vi så konkludere om det er lykkes at opfylde vores krav til det fuldente program resultat, som vi satte som mål i problemformuleringen.

\section{Problemformulering}

Der er mange forskellige strategier i kortspillet Røvhul. Ved at programmere forskellige AI med forskellige strategier kan vi spille dem mod hinanden og finde den bedste strategi.
\\
For at kunne dette skal vi først lave en implementering af Røvhul og derefter skal der identificeres og implementeres strategier. Vi forventer at vi med denne metode kan lave en AI der kan vinde over mennesker.

\vfill
\pagebreak

\section{Dokumentation}
Selve kortspils delen af projektet er lavet i Rust, hvor af spillets regler også er implementeret. For at undgå at de forskellige AI's overholder reglerne, er der også blevet implementeret en funktion som skal tjekke om de forskelle træk som bliver lavet er lovlige. Og hvis de ikke er vil spil delen bede den specifikke AI om at lave et nyt træk.
Vi har valgt at lave de forskellige AI's i Rust og C++,

\begin{figure}[H]
	\centering
	% Graphic for TeX using PGF
% Title: /home/marcus/Desktop/Diagram1.dia
% Creator: Dia v0.97.3
% CreationDate: Tue Mar 26 09:10:08 2019
% For: marcus
% \usepackage{tikz}
% The following commands are not supported in PSTricks at present
% We define them conditionally, so when they are implemented,
% this pgf file will use them.
\ifx\du\undefined
  \newlength{\du}
\fi
\setlength{\du}{15\unitlength}
\begin{tikzpicture}
\pgftransformxscale{1.000000}
\pgftransformyscale{-1.000000}
\definecolor{dialinecolor}{rgb}{0.000000, 0.000000, 0.000000}
\pgfsetstrokecolor{dialinecolor}
\definecolor{dialinecolor}{rgb}{1.000000, 1.000000, 1.000000}
\pgfsetfillcolor{dialinecolor}
\pgfsetlinewidth{0.100000\du}
\pgfsetdash{}{0pt}
\pgfsetdash{}{0pt}
\pgfsetbuttcap
\pgfsetmiterjoin
\pgfsetlinewidth{0.100000\du}
\pgfsetbuttcap
\pgfsetmiterjoin
\pgfsetdash{}{0pt}
\definecolor{dialinecolor}{rgb}{1.000000, 1.000000, 1.000000}
\pgfsetfillcolor{dialinecolor}
\pgfpathmoveto{\pgfpoint{4.516667\du}{0.000000\du}}
\pgfpathlineto{\pgfpoint{9.783333\du}{0.000000\du}}
\pgfpathcurveto{\pgfpoint{10.510509\du}{0.000000\du}}{\pgfpoint{11.100000\du}{0.447715\du}}{\pgfpoint{11.100000\du}{1.000000\du}}
\pgfpathcurveto{\pgfpoint{11.100000\du}{1.552285\du}}{\pgfpoint{10.510509\du}{2.000000\du}}{\pgfpoint{9.783333\du}{2.000000\du}}
\pgfpathlineto{\pgfpoint{4.516667\du}{2.000000\du}}
\pgfpathcurveto{\pgfpoint{3.789491\du}{2.000000\du}}{\pgfpoint{3.200000\du}{1.552285\du}}{\pgfpoint{3.200000\du}{1.000000\du}}
\pgfpathcurveto{\pgfpoint{3.200000\du}{0.447715\du}}{\pgfpoint{3.789491\du}{0.000000\du}}{\pgfpoint{4.516667\du}{0.000000\du}}
\pgfusepath{fill}
\definecolor{dialinecolor}{rgb}{0.000000, 0.000000, 0.000000}
\pgfsetstrokecolor{dialinecolor}
\pgfpathmoveto{\pgfpoint{4.516667\du}{0.000000\du}}
\pgfpathlineto{\pgfpoint{9.783333\du}{0.000000\du}}
\pgfpathcurveto{\pgfpoint{10.510509\du}{0.000000\du}}{\pgfpoint{11.100000\du}{0.447715\du}}{\pgfpoint{11.100000\du}{1.000000\du}}
\pgfpathcurveto{\pgfpoint{11.100000\du}{1.552285\du}}{\pgfpoint{10.510509\du}{2.000000\du}}{\pgfpoint{9.783333\du}{2.000000\du}}
\pgfpathlineto{\pgfpoint{4.516667\du}{2.000000\du}}
\pgfpathcurveto{\pgfpoint{3.789491\du}{2.000000\du}}{\pgfpoint{3.200000\du}{1.552285\du}}{\pgfpoint{3.200000\du}{1.000000\du}}
\pgfpathcurveto{\pgfpoint{3.200000\du}{0.447715\du}}{\pgfpoint{3.789491\du}{0.000000\du}}{\pgfpoint{4.516667\du}{0.000000\du}}
\pgfusepath{stroke}
% setfont left to latex
\definecolor{dialinecolor}{rgb}{0.000000, 0.000000, 0.000000}
\pgfsetstrokecolor{dialinecolor}
\node at (7.150000\du,1.120000\du){Start};
\definecolor{dialinecolor}{rgb}{1.000000, 1.000000, 1.000000}
\pgfsetfillcolor{dialinecolor}
\fill (3.150000\du,4.050000\du)--(3.150000\du,5.950000\du)--(11.050000\du,5.950000\du)--(11.050000\du,4.050000\du)--cycle;
\pgfsetlinewidth{0.100000\du}
\pgfsetdash{}{0pt}
\pgfsetdash{}{0pt}
\pgfsetmiterjoin
\definecolor{dialinecolor}{rgb}{0.000000, 0.000000, 0.000000}
\pgfsetstrokecolor{dialinecolor}
\draw (3.150000\du,4.050000\du)--(3.150000\du,5.950000\du)--(11.050000\du,5.950000\du)--(11.050000\du,4.050000\du)--cycle;
% setfont left to latex
\definecolor{dialinecolor}{rgb}{0.000000, 0.000000, 0.000000}
\pgfsetstrokecolor{dialinecolor}
\node at (7.100000\du,5.107500\du){Lav et sæt spillekort};
\definecolor{dialinecolor}{rgb}{1.000000, 1.000000, 1.000000}
\pgfsetfillcolor{dialinecolor}
\fill (3.150000\du,8.950000\du)--(3.150000\du,10.850000\du)--(11.150000\du,10.850000\du)--(11.150000\du,8.950000\du)--cycle;
\pgfsetlinewidth{0.100000\du}
\pgfsetdash{}{0pt}
\pgfsetdash{}{0pt}
\pgfsetmiterjoin
\definecolor{dialinecolor}{rgb}{0.000000, 0.000000, 0.000000}
\pgfsetstrokecolor{dialinecolor}
\draw (3.150000\du,8.950000\du)--(3.150000\du,10.850000\du)--(11.150000\du,10.850000\du)--(11.150000\du,8.950000\du)--cycle;
% setfont left to latex
\definecolor{dialinecolor}{rgb}{0.000000, 0.000000, 0.000000}
\pgfsetstrokecolor{dialinecolor}
\node at (7.150000\du,10.007500\du){Bland spillekortene};
\definecolor{dialinecolor}{rgb}{1.000000, 1.000000, 1.000000}
\pgfsetfillcolor{dialinecolor}
\fill (3.200000\du,13.250000\du)--(3.200000\du,15.950000\du)--(11.000000\du,15.950000\du)--(11.000000\du,13.250000\du)--cycle;
\pgfsetlinewidth{0.100000\du}
\pgfsetdash{}{0pt}
\pgfsetdash{}{0pt}
\pgfsetmiterjoin
\definecolor{dialinecolor}{rgb}{0.000000, 0.000000, 0.000000}
\pgfsetstrokecolor{dialinecolor}
\draw (3.200000\du,13.250000\du)--(3.200000\du,15.950000\du)--(11.000000\du,15.950000\du)--(11.000000\du,13.250000\du)--cycle;
% setfont left to latex
\definecolor{dialinecolor}{rgb}{0.000000, 0.000000, 0.000000}
\pgfsetstrokecolor{dialinecolor}
\node at (7.100000\du,14.395000\du){Fordel hver hånd};
% setfont left to latex
\definecolor{dialinecolor}{rgb}{0.000000, 0.000000, 0.000000}
\pgfsetstrokecolor{dialinecolor}
\node at (7.100000\du,15.195000\du){med en spiller};
\definecolor{dialinecolor}{rgb}{1.000000, 1.000000, 1.000000}
\pgfsetfillcolor{dialinecolor}
\fill (7.101144\du,22.088400\du)--(12.506289\du,25.977221\du)--(7.101144\du,29.866042\du)--(1.696000\du,25.977221\du)--cycle;
\pgfsetlinewidth{0.100000\du}
\pgfsetdash{}{0pt}
\pgfsetdash{}{0pt}
\pgfsetmiterjoin
\definecolor{dialinecolor}{rgb}{0.000000, 0.000000, 0.000000}
\pgfsetstrokecolor{dialinecolor}
\draw (7.101144\du,22.088400\du)--(12.506289\du,25.977221\du)--(7.101144\du,29.866042\du)--(1.696000\du,25.977221\du)--cycle;
% setfont left to latex
\definecolor{dialinecolor}{rgb}{0.000000, 0.000000, 0.000000}
\pgfsetstrokecolor{dialinecolor}
\node at (7.101144\du,25.772221\du){Er der kort tilbage};
% setfont left to latex
\definecolor{dialinecolor}{rgb}{0.000000, 0.000000, 0.000000}
\pgfsetstrokecolor{dialinecolor}
\node at (7.101144\du,26.572221\du){i spillerens hånd};
\pgfsetlinewidth{0.100000\du}
\pgfsetdash{}{0pt}
\pgfsetdash{}{0pt}
\pgfsetmiterjoin
\pgfsetbuttcap
{
\definecolor{dialinecolor}{rgb}{0.000000, 0.000000, 0.000000}
\pgfsetfillcolor{dialinecolor}
% was here!!!
\pgfsetarrowsend{stealth}
{\pgfsetcornersarced{\pgfpoint{0.000000\du}{0.000000\du}}\definecolor{dialinecolor}{rgb}{0.000000, 0.000000, 0.000000}
\pgfsetstrokecolor{dialinecolor}
\draw (1.696000\du,25.977221\du)--(1.696000\du,25.950000\du)--(0.350000\du,25.950000\du)--(0.350000\du,19.273282\du)--(5.393160\du,19.273282\du);
}}
% setfont left to latex
\definecolor{dialinecolor}{rgb}{0.000000, 0.000000, 0.000000}
\pgfsetstrokecolor{dialinecolor}
\node[anchor=west] at (2.350000\du,19.000000\du){Ja};
\definecolor{dialinecolor}{rgb}{1.000000, 1.000000, 1.000000}
\pgfsetfillcolor{dialinecolor}
\pgfpathellipse{\pgfpoint{7.150028\du}{19.273282\du}}{\pgfpoint{1.706728\du}{0\du}}{\pgfpoint{0\du}{1.676682\du}}
\pgfusepath{fill}
\pgfsetlinewidth{0.100000\du}
\pgfsetdash{}{0pt}
\pgfsetdash{}{0pt}
\pgfsetmiterjoin
\definecolor{dialinecolor}{rgb}{0.000000, 0.000000, 0.000000}
\pgfsetstrokecolor{dialinecolor}
\pgfpathellipse{\pgfpoint{7.150028\du}{19.273282\du}}{\pgfpoint{1.706728\du}{0\du}}{\pgfpoint{0\du}{1.676682\du}}
\pgfusepath{stroke}
% setfont left to latex
\definecolor{dialinecolor}{rgb}{0.000000, 0.000000, 0.000000}
\pgfsetstrokecolor{dialinecolor}
\node at (7.150028\du,19.468282\du){1};
\pgfsetlinewidth{0.100000\du}
\pgfsetdash{}{0pt}
\pgfsetdash{}{0pt}
\pgfsetbuttcap
{
\definecolor{dialinecolor}{rgb}{0.000000, 0.000000, 0.000000}
\pgfsetfillcolor{dialinecolor}
% was here!!!
\pgfsetarrowsend{stealth}
\definecolor{dialinecolor}{rgb}{0.000000, 0.000000, 0.000000}
\pgfsetstrokecolor{dialinecolor}
\draw (7.100000\du,5.950000\du)--(7.137347\du,8.900446\du);
}
\pgfsetlinewidth{0.100000\du}
\pgfsetdash{}{0pt}
\pgfsetdash{}{0pt}
\pgfsetbuttcap
{
\definecolor{dialinecolor}{rgb}{0.000000, 0.000000, 0.000000}
\pgfsetfillcolor{dialinecolor}
% was here!!!
\pgfsetarrowsend{stealth}
\definecolor{dialinecolor}{rgb}{0.000000, 0.000000, 0.000000}
\pgfsetstrokecolor{dialinecolor}
\draw (7.150000\du,2.000000\du)--(7.100000\du,4.050000\du);
}
\pgfsetlinewidth{0.100000\du}
\pgfsetdash{}{0pt}
\pgfsetdash{}{0pt}
\pgfsetbuttcap
{
\definecolor{dialinecolor}{rgb}{0.000000, 0.000000, 0.000000}
\pgfsetfillcolor{dialinecolor}
% was here!!!
\pgfsetarrowsend{stealth}
\definecolor{dialinecolor}{rgb}{0.000000, 0.000000, 0.000000}
\pgfsetstrokecolor{dialinecolor}
\draw (7.150000\du,10.850000\du)--(7.118665\du,13.200159\du);
}
\pgfsetlinewidth{0.100000\du}
\pgfsetdash{}{0pt}
\pgfsetdash{}{0pt}
\pgfsetbuttcap
{
\definecolor{dialinecolor}{rgb}{0.000000, 0.000000, 0.000000}
\pgfsetfillcolor{dialinecolor}
% was here!!!
\pgfsetarrowsend{stealth}
\definecolor{dialinecolor}{rgb}{0.000000, 0.000000, 0.000000}
\pgfsetstrokecolor{dialinecolor}
\draw (7.114989\du,16.000216\du)--(7.131548\du,17.547043\du);
}
\pgfsetlinewidth{0.100000\du}
\pgfsetdash{}{0pt}
\pgfsetdash{}{0pt}
\pgfsetbuttcap
{
\definecolor{dialinecolor}{rgb}{0.000000, 0.000000, 0.000000}
\pgfsetfillcolor{dialinecolor}
% was here!!!
\pgfsetarrowsend{stealth}
\definecolor{dialinecolor}{rgb}{0.000000, 0.000000, 0.000000}
\pgfsetstrokecolor{dialinecolor}
\draw (7.150028\du,20.949964\du)--(7.101144\du,22.088400\du);
}
\end{tikzpicture}

	\caption{Diagram 1}
\end{figure}

\begin{figure}[H]
	\centering
	% Graphic for TeX using PGF
% Title: /home/marcus/Diagram1.dia
% Creator: Dia v0.97.3
% CreationDate: Tue Mar 26 09:08:59 2019
% For: marcus
% \usepackage{tikz}
% The following commands are not supported in PSTricks at present
% We define them conditionally, so when they are implemented,
% this pgf file will use them.
\ifx\du\undefined
  \newlength{\du}
\fi
\setlength{\du}{15\unitlength}
\begin{tikzpicture}
\pgftransformxscale{1.000000}
\pgftransformyscale{-1.000000}
\definecolor{dialinecolor}{rgb}{0.000000, 0.000000, 0.000000}
\pgfsetstrokecolor{dialinecolor}
\definecolor{dialinecolor}{rgb}{1.000000, 1.000000, 1.000000}
\pgfsetfillcolor{dialinecolor}
\definecolor{dialinecolor}{rgb}{1.000000, 1.000000, 1.000000}
\pgfsetfillcolor{dialinecolor}
\fill (2.580000\du,7.603360\du)--(2.580000\du,9.503360\du)--(13.290000\du,9.503360\du)--(13.290000\du,7.603360\du)--cycle;
\pgfsetlinewidth{0.100000\du}
\pgfsetdash{}{0pt}
\pgfsetdash{}{0pt}
\pgfsetmiterjoin
\definecolor{dialinecolor}{rgb}{0.000000, 0.000000, 0.000000}
\pgfsetstrokecolor{dialinecolor}
\draw (2.580000\du,7.603360\du)--(2.580000\du,9.503360\du)--(13.290000\du,9.503360\du)--(13.290000\du,7.603360\du)--cycle;
% setfont left to latex
\definecolor{dialinecolor}{rgb}{0.000000, 0.000000, 0.000000}
\pgfsetstrokecolor{dialinecolor}
\node at (7.935000\du,8.748360\du){Lad næste spiller spille sin tur};
\definecolor{dialinecolor}{rgb}{1.000000, 1.000000, 1.000000}
\pgfsetfillcolor{dialinecolor}
\pgfpathellipse{\pgfpoint{7.831664\du}{3.501682\du}}{\pgfpoint{2.103364\du}{0\du}}{\pgfpoint{0\du}{2.001682\du}}
\pgfusepath{fill}
\pgfsetlinewidth{0.100000\du}
\pgfsetdash{}{0pt}
\pgfsetdash{}{0pt}
\pgfsetmiterjoin
\definecolor{dialinecolor}{rgb}{0.000000, 0.000000, 0.000000}
\pgfsetstrokecolor{dialinecolor}
\pgfpathellipse{\pgfpoint{7.831664\du}{3.501682\du}}{\pgfpoint{2.103364\du}{0\du}}{\pgfpoint{0\du}{2.001682\du}}
\pgfusepath{stroke}
% setfont left to latex
\definecolor{dialinecolor}{rgb}{0.000000, 0.000000, 0.000000}
\pgfsetstrokecolor{dialinecolor}
\node at (7.831664\du,3.696682\du){1};
\pgfsetlinewidth{0.100000\du}
\pgfsetdash{}{0pt}
\pgfsetdash{}{0pt}
\pgfsetbuttcap
{
\definecolor{dialinecolor}{rgb}{0.000000, 0.000000, 0.000000}
\pgfsetfillcolor{dialinecolor}
% was here!!!
\pgfsetarrowsend{stealth}
\definecolor{dialinecolor}{rgb}{0.000000, 0.000000, 0.000000}
\pgfsetstrokecolor{dialinecolor}
\draw (7.831664\du,5.503364\du)--(7.901900\du,7.576408\du);
}
\definecolor{dialinecolor}{rgb}{1.000000, 1.000000, 1.000000}
\pgfsetfillcolor{dialinecolor}
\fill (7.712047\du,11.237960\du)--(12.516995\du,14.548189\du)--(7.712047\du,17.858419\du)--(2.907100\du,14.548189\du)--cycle;
\pgfsetlinewidth{0.100000\du}
\pgfsetdash{}{0pt}
\pgfsetdash{}{0pt}
\pgfsetmiterjoin
\definecolor{dialinecolor}{rgb}{0.000000, 0.000000, 0.000000}
\pgfsetstrokecolor{dialinecolor}
\draw (7.712047\du,11.237960\du)--(12.516995\du,14.548189\du)--(7.712047\du,17.858419\du)--(2.907100\du,14.548189\du)--cycle;
% setfont left to latex
\definecolor{dialinecolor}{rgb}{0.000000, 0.000000, 0.000000}
\pgfsetstrokecolor{dialinecolor}
\node at (7.712047\du,14.743189\du){Er det et validt træk};
\pgfsetlinewidth{0.100000\du}
\pgfsetdash{}{0pt}
\pgfsetdash{}{0pt}
\pgfsetbuttcap
{
\definecolor{dialinecolor}{rgb}{0.000000, 0.000000, 0.000000}
\pgfsetfillcolor{dialinecolor}
% was here!!!
\pgfsetarrowsend{stealth}
\definecolor{dialinecolor}{rgb}{0.000000, 0.000000, 0.000000}
\pgfsetstrokecolor{dialinecolor}
\draw (7.897809\du,9.553352\du)--(7.833893\du,11.271962\du);
}
\definecolor{dialinecolor}{rgb}{1.000000, 1.000000, 1.000000}
\pgfsetfillcolor{dialinecolor}
\fill (16.085000\du,13.253360\du)--(16.085000\du,15.953360\du)--(23.285000\du,15.953360\du)--(23.285000\du,13.253360\du)--cycle;
\pgfsetlinewidth{0.100000\du}
\pgfsetdash{}{0pt}
\pgfsetdash{}{0pt}
\pgfsetmiterjoin
\definecolor{dialinecolor}{rgb}{0.000000, 0.000000, 0.000000}
\pgfsetstrokecolor{dialinecolor}
\draw (16.085000\du,13.253360\du)--(16.085000\du,15.953360\du)--(23.285000\du,15.953360\du)--(23.285000\du,13.253360\du)--cycle;
% setfont left to latex
\definecolor{dialinecolor}{rgb}{0.000000, 0.000000, 0.000000}
\pgfsetstrokecolor{dialinecolor}
\node at (19.685000\du,14.798360\du){Lav et nyt træk};
\pgfsetlinewidth{0.100000\du}
\pgfsetdash{}{0pt}
\pgfsetdash{}{0pt}
\pgfsetbuttcap
{
\definecolor{dialinecolor}{rgb}{0.000000, 0.000000, 0.000000}
\pgfsetfillcolor{dialinecolor}
% was here!!!
\pgfsetarrowsend{stealth}
\definecolor{dialinecolor}{rgb}{0.000000, 0.000000, 0.000000}
\pgfsetstrokecolor{dialinecolor}
\draw (12.555598\du,14.570508\du)--(16.035529\du,14.586543\du);
}
% setfont left to latex
\definecolor{dialinecolor}{rgb}{0.000000, 0.000000, 0.000000}
\pgfsetstrokecolor{dialinecolor}
\node[anchor=west] at (13.385000\du,14.303360\du){Nej};
\pgfsetlinewidth{0.100000\du}
\pgfsetdash{}{0pt}
\pgfsetdash{}{0pt}
\pgfsetbuttcap
{
\definecolor{dialinecolor}{rgb}{0.000000, 0.000000, 0.000000}
\pgfsetfillcolor{dialinecolor}
% was here!!!
\pgfsetarrowsend{stealth}
\definecolor{dialinecolor}{rgb}{0.000000, 0.000000, 0.000000}
\pgfsetstrokecolor{dialinecolor}
\draw (7.712047\du,17.858419\du)--(7.735000\du,20.203360\du);
}
\pgfsetlinewidth{0.100000\du}
\pgfsetdash{}{0pt}
\pgfsetdash{}{0pt}
\pgfsetmiterjoin
\pgfsetbuttcap
{
\definecolor{dialinecolor}{rgb}{0.000000, 0.000000, 0.000000}
\pgfsetfillcolor{dialinecolor}
% was here!!!
\pgfsetarrowsend{stealth}
{\pgfsetcornersarced{\pgfpoint{0.000000\du}{0.000000\du}}\definecolor{dialinecolor}{rgb}{0.000000, 0.000000, 0.000000}
\pgfsetstrokecolor{dialinecolor}
\draw (19.685000\du,13.253360\du)--(19.685000\du,11.553360\du)--(10.114521\du,11.553360\du)--(10.114521\du,12.893075\du);
}}
\end{tikzpicture}

	\caption{Diagram 2}
\end{figure}

Diagram 1, er en visualisering over selve spillet samt loopet, som tjekker om der er flere kort der kan spilles.
\bigbreak

Diagram 2, er en visualisering over det loop som står for at næste spiller får sin tur, samt at trækkene er valide.

\vfill
\pagebreak


\subsection{Reglerne}
Reglerne i røvhul er forholdsvist simple, men de skaber muglighed for et rigt og advanceret strategisk kortspil.
De forskellige regler betyder at forskellige strategier kan have sine fordel i forskellige scenarier. 
Der er et par forskellige reglsæt og der kan til tider være uenigheder om enkelte regler. Vi har derfor defineret det reglset som vi har lavet spillet ud fra.

\begin{itemize}
	\item Målet i røvhul er at spille alle sine kort så man har nul kort tilbage i hånden.
	\item Den første person som har spillet alle sine kort og vinder spillet bliver kåret præsident. Den spiller som sidst har kort i hånden bliver kåret røvhul. Yderligere hvis mere end 3 spillere spiller på en gang bliver den anden første og anden sidste henholdsvist kåret visepræsident og viserøvhul.
	\item I første spil er det spilleren med klør tre som starter, dog behøver de ikke spille klør tre i første runde. I følgende spil er det røvhullet fra det tidligere spil som starter.
	\item Hver spiller skiftes til at ligge et kort ned på toppen af bunken i urets retning. Kortet man ligger ned skal altid være lig med eller højere end det øverste kort i bunken. 
	\item Det højeste kort er to efterfulgt af es, konge osv. ned til tre. Det gælder dog ikke 10 som ryder bunken og derfor er højere end to.
	\item Når bunken rydes kasseres alle kort i bunken og spillet fortsætter. Det er altid den spiller som sidst har lagt et kort i bunken som starter med at ligge et kort.
	\item Hvis der ikke er nogle kort i bunken må man spille et hvilket som helst kort. Yderligere er det tiladt at ligge to eller flere af det samme kort eks. to konger eller tre es. Hvis to eller flere kort bliver lagt i bunken på denne måde skal alle følgende spillere ligge den samme mængde kort på bunken som også er ens og er lig med eller højere.
	\item Hvis to spiller ligger to ens kort i streg er det ikke tiladt at den tredje spiller ligger to ens kort i bunken.
	\item Hvis der på noget tidspunkt ligger fire ens kort i streg i bunken rydes bunken.
	\item Hvis man ikke kan, eller ikke vil spille et kort er det tiladt at sige pas, men hvis en spiller siger pas og bliver sprunget over vil de automatisk blive sprunget over ind til den nuværende bunke rydes.
	\item Når man begynder et spil efter det første spil eller efterfulgte spil er afsluttede skal røvhullet give sine to bedste/højeste kort hvor ti er højest så to, es osv til præsidenten. Præsidenten skal derefter give to af sine kort til røvhullet og præsidenten må selv vælge hvilke to kort de giver. Alle kort gives med bagsiden opad. Hvis man spiller visepræsident og -røvhul gælder samme regl for dem, dog bytter de kun et kort. 
\end{itemize}

Vi har analyseret regelerne og udvalgt enkelte regeler som vi føler ikke vil passe ind i en digitalisering af spiller. For det første giver det ikke mening at rækkefølgen af spillere er givet ud fra hvordan man sidder rundt om et bord. Derfor har vi i valgt at tilfældigt lave rækkefølgen før spillet starter og derved simulere ture rundt om et bord. Da der kun er tre spillere med i spillet spiller vi ikke med visepræsident og -røvhul reglen.

\subsection{Rust}


\begin{minted}{rust}
	fn main() {
		println!("Jeg er fart")
	}
\end{minted}


\subsection{C++}



\vfill
\pagebreak

\section{Konklusion}

\vfill
\pagebreak

\end{document}